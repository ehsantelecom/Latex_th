%%%%%%%%%%%%%%%%%%%%%%%%%%%%%%%%%%%%%%%%%%%%%%%%%%%% 1.o Esempio con la classe toptesi
\documentclass[12pt,oneside,cucitura]{toptesi}
%\documentclass[twoside,cucitura,pdfa]{toptesi}
%%%%%%%%%%%%%%%%%%%%%%%%%%%%%%%%%%%%%%%%%%%%%%%%%%%% 2.o Esempio con la classe toptesi
% \documentclass[11pt,twoside,oldstyle,autoretitolo,classica,greek]{toptesi}
% \usepackage[or]{teubner}
%%%%%%%%%%%%%%%%%%%%%%%%%%%%%%%%%%%%%%%%%%%%%%%%%%%%
% Commentare la riga seguente se si   specificata l'opzione "pdfa"
\usepackage{hyperref}
\hypersetup{%
    pdfpagemode={UseOutlines},
    bookmarksopen,
    pdfstartview={FitH},
    colorlinks,
    linkcolor={blue},
    citecolor={red},
    urlcolor={blue}
  }

\usepackage{toptesi}

%\usepackage{times}
\usepackage{algorithm}
\usepackage{algorithmic}
%\usepackage{algorithmicx}

\usepackage{acronym}
\usepackage{xspace}

\usepackage{graphicx}
\usepackage{epstopdf}
\usepackage{multirow}
\usepackage{amssymb, amsmath}
\usepackage{pdfpages}
%\usepackage{enumerate}
\usepackage{url}
%\usepackage{natbib}
%\usepackage{ecrc}
\usepackage{subfigure}
\usepackage{dblfloatfix}
\usepackage{epsfig}
\usepackage{color}
\usepackage{fancyvrb}

\DefineVerbatimEnvironment{code}{Verbatim}{fontsize=\tiny}

%\usepackage{algpseudocode}

\usepackage{paralist}
\usepackage{ctable}
\usepackage{rotating}
\usepackage{cite}
\usepackage{appendix}
\usepackage[latin1]{inputenc}
\usepackage[english]{babel}
\input{kvmacros}


\usepackage{framedbox}
\usepackage{subfigure}
\usepackage{slashbox}
\usepackage{pifont}
\usepackage[multiple]{footmisc}

\usepackage[T1]{fontenc}
\usepackage{times}%
\usepackage{topcapt}
\usepackage{natbib}% for bibliography sorting/compressing
\usepackage{pdfpages}
%\usepackage{amsmath}
%\usepackage{endnotes}
\usepackage{graphics}


%\usepackage{times}
\newpage\newpage\newcounter{eqn}
\renewcommand*{\theeqn}{\alph{eqn})}
\newcommand{\num}{\refstepcounter{eqn}\text{\theeqn}\;}

\makeatletter
%\newcommand{\putindeepbox}[2][0.7\baselineskip]{{%
\newcommand{\putindeepbox}[2][scale=0.4]{{%
    \setbox0=\hbox{#2}%
    \setbox0=\vbox{\noindent\hsize=\wd0\unhbox0}
    \@tempdima=\dp0
    \advance\@tempdima by \ht0
    \advance\@tempdima by -#1\relax
    \dp0=\@tempdima
    \ht0=#1\relax
    \box0
}}
\makeatother




\usepackage{multirow}
\usepackage{amssymb, amsmath}
%\usepackage{enumerate}
%\usepackage{url}
%\usepackage{color} 
\usepackage{listings}
%\usepackage{paralist}
%\usepackage{ctable}

\usepackage{lmodern}



	\ateneo{POLITECNICO DI TORINO}
	\facolta{Elettronica e Telecomunicazioni}
 	\titolo{Design and Implementation of OFDM System on FPGA}
 	\sottotitolo {}
 	\corsodilaurea{Electronic and Telecommunication Engineering}
 	
 	\candidato{Seyed-Ehsan Koohestani}
 	\coordinatore{Prof. Marina Mondin (PoliTo)}
 	\secondorelatore{Prof. Roberto Garello (PoliTo)}
 	\terzorelatore{Prof. Amir Keyvan Khandani (uWaterloo)}
 	\tutore{Prof. Marina Mondin - Prof. Roberto Garello - Prof. Amir Keyvan Khandani}
 	 	
 	\esamedidottorato{March 2015}
 	%\ciclodidottorato{XXVI Ciclo}
 	
 	\logosede{logopolito.eps}
 	
\newtheorem{osservazione}{Osservazione}% Standard LaTeX
 	


\newcommand{\figref}[1] {Fig.~\ref{#1}}
\newcommand{\secref}[1] {Section~\ref{#1}}
\newcommand{\tabref}[1] {Table~\ref{#1}}
\newcommand{\eqtnref}[1] {Eq.~\ref{#1}}
\newcommand{\algref}[1] {Alg.~\ref{#1}}

\newcommand{\refnet}{base network\xspace}
\newcommand{\germany}{Germany17\xspace}
\newcommand{\abilene}{Abilene\xspace}
\newcommand{\geant}{G\'{e}ant\xspace}

\newcommand{\myparagraph}[1]{\par{\textbf{#1}}}

\newcommand{\nobel} {{\sc Nobel}\xspace}
\newcommand{\zimpl} {{\sc Zimpl}\xspace}
\newcommand{\cplex} {{\sc Cplex}\xspace}
%% ----------------------------------------------------------------------
%%    general math
%% ----------------------------------------------------------------------
\renewcommand{\implies}{\ensuremath{\Rightarrow}}
\newcommand{\R}            {\ensuremath{\mathbb{R}}\xspace}
\newcommand{\Rplus}        {\ensuremath{\R_+}\xspace}
\newcommand{\Z}            {\ensuremath{\mathbb{Z}}\xspace}
\newcommand{\Zplus}        {\ensuremath{\Z_+}\xspace}
\newcommand{\card}[1]      {\ensuremath{\left|{#1}\right|}\xspace}
\newcommand{\abs}[1]       {\ensuremath{\card{#1}}\xspace}
% \newcommand{\norm}[1]      {\ensuremath{\left\|{#1}\right\|}\xspace}
\newcommand{\NP}{NP\xspace}
\newcommand{\NPhard}{\NP -hard\xspace}

\renewcommand {\atop} [2] {\genfrac{}{}{0pt}{}{#1}{#2}}

% sets
\newcommand{\set}[1] {\left\{ {#1} \right\}}
%\newcommand{\settype}      [1] {\ensuremath{\mathcal{#1}}\xspace}
\newcommand{\settype}      [1] {\ensuremath{#1}\xspace}
\newcommand{\elementtype}  [1] {\ensuremath{#1}\xspace}

% nodes and edges
\newcommand{\node}             {\elementtype{i}}
\newcommand{\othernode}        {\elementtype{j}}
\newcommand{\physgraph}        {\settype{G}}
\newcommand{\virtgraph}        {\settype{H}}
\newcommand{\nodes}            {\settype{V}}
\newcommand{\allnodepairs}     {\settype{\nodes\times\nodes}}
\newcommand{\physlink}         {\elementtype{e}}
\newcommand{\allphyslinks}     {\settype{E}}
\newcommand{\virtlink}         {\elementtype{l}}
%\newcommand{\allvirtlinks}     {\settype{L}}
\newcommand{\link}          {\elementtype{p}}
\newcommand{\alllinks}          {\elementtype{P}}
\newcommand{\alllinksnodepair}          {\elementtype{\alllinks_{\nodepair}}}
\newcommand{\alllinksending}[1]          {\alllinks_#1}
\newcommand{\allphyslinksending}[1]          {\delta_{\allphyslinks}(#1)}
\newcommand{\alllinksusing}[1]          {\link\in\alllinks:\,#1\in\link}
\newcommand{\alllinkstraversing}[1]          {\alllinks_#1}
\newcommand{\nodepair}          {\elementtype{(\node,\othernode)}}
\newcommand{\linkfwd}          {\elementtype{ij}}
\newcommand{\linkbwd}          {\elementtype{ji}}

\newcommand{\linkweight}          {\elementtype{w_{\virtlink}}}
\newcommand{\demandsrcnode}             {\elementtype{a}}
\newcommand{\demandtrgtnode}             {\elementtype{b}}
\newcommand{\demandsrctrgtnodepair}          {\elementtype{(\demandsrcnode,\demandtrgtnode)}}
\newcommand{\physlinklength}          {\elementtype{L_{\physlink}}}
\newcommand{\physlinkamplifiers}          {\elementtype{N_{a}^{\physlink}}}

% time periods
\newcommand{\period}       {\ensuremath{t}\xspace}
\newcommand{\allPeriods}   {\ensuremath{T}\xspace}
\newcommand{\tinT}         {\ensuremath{\period \in \allPeriods}\xspace}

% demands and commodities
\newcommand{\com}                   {\elementtype{k}}
\newcommand{\allcoms}               {\settype{K}}

% demand values
\newcommand{\demandvalueof} [1]{\ensuremath{d_{#1}}\xspace}
\newcommand{\demandvalueofat} [2]{\ensuremath{d_{#1}}\xspace}
\newcommand{\dij}            {\ensuremath{d_{\node\othernode}}\xspace}
\newcommand{\dki}         {\ensuremath{d_{\com\node}}\xspace}
\newcommand{\demandvalue}    {\settype{\demandvalueof{\demand}}}
\newcommand{\comdemandvalue}   {\ensuremath{\demandvalueof{}^{\com}}\xspace}
\newcommand{\nodedemandvalue}  {\ensuremath{\demandvalueof{\node}^{\com}}\xspace}
\newcommand{\nodeemanatingdemandvalue}  {\ensuremath{\demandvalueof{\node}}\xspace}

\newcommand{\dji}            {\ensuremath{d_{\othernode\node}}\xspace}
\newcommand{\dab}            {\ensuremath{d_{\demandsrcnode\demandtrgtnode}}\xspace}
\newcommand{\dba}            {\ensuremath{d_{\demandtrgtnode\demandsrcnode}}\xspace}

% iterators
\newcommand{\iinV}         {\ensuremath{\node     \in \nodes}\xspace}
\newcommand{\ijinVV}         {\ensuremath{\nodepair \in \allnodepairs}\xspace}
\newcommand{\abinVV}         {\ensuremath{\demandsrctrgtnodepair \in \allnodepairs}\xspace}
\newcommand{\einE}         {\ensuremath{\physlink \in \allphyslinks}\xspace}
\newcommand{\pinP}         {\ensuremath{\link  \in \alllinks}\xspace}
\newcommand{\pinPe}         {\ensuremath{\link\in\alllinkstraversing{\physlink}}\xspace}
\newcommand{\pinPij}         {\ensuremath{\link \in \alllinksnodepair}\xspace}
%\newcommand{\linL}         {\ensuremath{\virtlink \in \allvirtlinks}\xspace}

\newcommand{\kinK}         {\ensuremath{\com      \in \allcoms}\xspace}
\newcommand{\kinKp}        {\ensuremath{\com      \in \allprotectedcoms}\xspace}
\newcommand{\ninN}        {\ensuremath{\router \in \allrouters}\xspace}
\newcommand{\oinO}        {\ensuremath{\oxc \in \alloxcs}\xspace}

% capacity modules
\newcommand{\router}       {\elementtype{n}}
\newcommand{\allrouters}   {\ensuremath{\settype{N}}\xspace}
% \newcommand{\linkmodule}    {\elementtype{m}}
% \newcommand{\allmodulesoflink}{\ensuremath{\settype{M}_{\link}}\xspace}
% \newcommand{\alllinkmodules}       {\ensuremath{\settype{M}}\xspace}

% capacities
\newcommand{\capacity}{\ensuremath{C}\xspace}
\newcommand{\caprouter}         {\ensuremath{R^{\router}}\xspace}
\newcommand{\capfiber}     {\ensuremath{B}\xspace}

% maximum admissible utilization (of a lightpath, logical
% link and bandwidth installed on a physical path)
\newcommand{\maxutil}     {\ensuremath{\delta}\xspace}

% cost
% \newcommand{\cost}          [2]  {\ensuremath{\kappa_{#1}^{#2}}\xspace}
% \newcommand{\costoxc}     {\ensuremath{\delta}\xspace}
\newcommand{\costrouter}      {\ensuremath{\alpha^{\router}}\xspace}
\newcommand{\costfiber} {\ensuremath{\beta^{\physlink}}\xspace}
% \newcommand{\costlinkmodule}  {\ensuremath{\cost{\link}{\linkmodule}}\xspace}
\newcommand{\costlinkmodule} {\ensuremath{\gamma}\xspace}
\newcommand{\costamplifier}      {\ensuremath{\beta^{a}}\xspace}
\newcommand{\costterminal}      {\ensuremath{\beta^{t}}\xspace}

% primal variables
\newcommand{\vartotalflownodepair}  {\ensuremath{f_{(\nodepair)}}\xspace}
\newcommand{\varflownodepairfwddemand}    {\ensuremath{f_{\node\othernode}^{\demandsrcnode\demandtrgtnode}}\xspace}
\newcommand{\varflownodepairbwddemand}    {\ensuremath{f_{\othernode\node}^{\demandsrcnode\demandtrgtnode}}\xspace}
\newcommand{\varflownodepairfwdcom}       {\ensuremath{f_{\node\othernode}^{\com}}\xspace}
\newcommand{\varflownodepairbwdcom}       {\ensuremath{f_{\othernode\node}^{\com}}\xspace}
\newcommand{\varflowlinkcom}        {\ensuremath{f_{\link}^{\com}}\xspace}
\newcommand{\varflowlink}           {\ensuremath{f_{\link}^{\nodepair}}\xspace}
\newcommand{\vartotalflowlink}      {\ensuremath{f_{\link}}\xspace}
\newcommand{\varcaplink}            {\ensuremath{y_{\link}}\xspace}
% \newcommand{\varcaplinkmodule}    {\ensuremath{y_{\link}^{\linkmodule}}\xspace}
\newcommand{\varcaprouter}         {\ensuremath{x_{\node}^{\router}}\xspace}
% \newcommand{\varcapoxc}       {\ensuremath{x_{\node}^{\oxc}}\xspace}
\newcommand{\varcapfiber}                    {\ensuremath{z_{\physlink}}\xspace}


\newcommand{\varflowvirtlink}           {\ensuremath{f_{\virtlink}^{\nodepair}}\xspace}
\newcommand{\varflownodepairaj}    {\ensuremath{f_{\demandsrcnode\othernode}^{\demandsrcnode\demandtrgtnode}}\xspace}
\newcommand{\varflownodepairbi}    {\ensuremath{f_{\node\demandtrgtnode}^{\demandsrcnode\demandtrgtnode}}\xspace}

%% ----------------------------------------------------------------------
%%    mathematical notation
%% ----------------------------------------------------------------------

%% for readability
\newcommand{\union}           {\cup}
\newcommand{\disjointunion}   {\uplus}
\newcommand{\intersection}    {\cap}
\newcommand{\bigunion}        {\bigcup}
\newcommand{\bigdisjointunion}{\biguplus}
\newcommand{\bigintersection} {\bigcap}
%\newcommand{\eps}             {\epsilon}


%% ----------------------------------------------------------------------
%%    GAGD notation
%% ----------------------------------------------------------------------

\newcommand{\maxnoimprovements}  {\ensuremath{\Delta}\xspace}
\newcommand{\populationsize}  {\ensuremath{\Theta}\xspace}
\newcommand{\offspringsize}  {\ensuremath{\Gamma}\xspace}
\newcommand{\population}  {\ensuremath{\Phi}\xspace}
\newcommand{\offspring}  {\ensuremath{\Psi}\xspace}

%---------------------------------------------------------------------
% End of file
%---------------------------------------------------------------------


\newcommand{\fmfi}[2][\empty]{\ifthenelse{\equal{#1}{\empty}}{\fixme[inline]{[FI] #2}}{\fixme[#1]{[FI] #2}}}
\newcommand{\fmlc}[2][\empty]{\ifthenelse{\equal{#1}{\empty}}{\fixme[inline]{[LC] #2}}{\fixme[#1]{[LC] #2}}}
\newcommand{\fmeb}[2][\empty]{\ifthenelse{\equal{#1}{\empty}}{\fixme[inline]{[EB] #2}}{\fixme[#1]{[EB] #2}}}
\newcommand{\todo}[1]{\par\noindent\fxwarning{#1}\par\noindent}
\newcommand{\error}[1]{\par\noindent\fxerror{#1}\par\noindent}
\newcommand{\fatal}[1]{\par\noindent\fixme{#1}\par\noindent}
\newcommand{\Remark}[1]{\par\noindent\fxnote{#1}\par\noindent}

\begin{document}
\english
\errorcontextlines=9
\setcleardoublepage{empty}

\expandafter\ifx\csname StileTrieste\endcsname\relax
    \frontespizio
\else
    \paginavuota
    \begin{dedica}
       To my family.

        \textdagger\ To my family.
    \end{dedica}
    \tomo
\fi

%\includepdf[pages={1}]{Cover.pdf}
\cleardoublepage

\sommario
\textit{The orthogonal frequency division multiplexing} (OFDM) technology provides a high transmission data rate in wireless and mobile communications where multipath fading is a severe issue in degradation of the quality. Managing feasible coherent bandwidth to overcome Inter-Symbol Interference, OFDM enhances communication performance at a relatively small bandwidth cost. The improvement can be reached by interactive proper channel estimation and compensation which needs synchronization of transmitter and receiver. A \textit{Discrete Fourier Transform} (DFT) algorithm- based configuration simplified the digital implementation of OFDM system on field programmable gate array (FPGA) as a highly flexible solution, which provide prominent performance.\\
In this thesis, steps to design a base-band OFDM system with channel estimation and timing synchronization upto implemented on FPGA are studied. It is a prototype based on the \textit{IEEE 802.11a} standard and the signals is transmitted and received using a bandwidth of 20 MHz. Focusing on the quadrature phase shift keying (QPSK) modulation, the system can achieve a throughput of $24 Mbps$. For the coarse estimation of timing, a modified maximum-normalized correlation (MNC) scheme is investigated and implemented. Starting from theoretical study, this thesis in detail describes the system design and verification on the basis of both MATLAB simulation and hardware implementation. Bit error rate (BER) verses bit energy to noise spectral density ($E_{b}/N_{0}$) is presented in the case of different channels. In the meanwhile, comparison is made between the simulation and implementation results, which verifies system performance from the system level to the register transfer level (RTL).\\
First of all, the entire system is modeled in MATLAB and a floating-point model is established. Then, the fixed-point model is created with the help of Xilinx$'$s System Generator for DSP (XSG) and Simulink. Subsequently, the system is synthesized and implemented within Xilinx$'$s Integrated Software Environment (ISE) tools and targeted to Xilinx Zynq board. What is more, a hardware co-simulation is devised to reduce the processing time while calculating the BER for the fixed-point model.\\
Some time-based standards on IEEE 802.11a are discussed and optimum implementation of on FPGA, for instance Cross-Correlation and algebraic machine, will be introduced. Besides, we will demonstrate an engineering steps for choosing the radio board and the processor software implementations.\\


\ringraziamenti
I am using this opportunity to express my gratitude to everyone who supported me throughout the course of this Master thesis. I am thankful for their aspiring guidance, invaluably constructive criticism and friendly advice during the project work. I am sincerely grateful to them for sharing their truthful and illuminating views on a number of issues related to the project.
I express my warm thanks to Prof. Amir Keyvan Khandani and CST Lab members for their support, trust and guidance at the University of Waterloo.
I would also like to thank guide of Prof. Roberto Garello and Prof. Marina Modin at Politecnico di Torino and their aspiration in Communication Systems.
Especially, I appreciate my family for their constant devotion and love during years of separation.\\

Thank you,

Author

\tablespagetrue\figurespagetrue % normalmente questa riga non serve ed e' commentata

\indici

\mainmatter



\include{introduction}
\section{Background}


High quality of services (QoS) and reaching to high data rate communication to overcome the necessities in multimedia services, telecommunications industry is working currently toward the forth generation (4G) wireless communication systems. The orthogonal frequency division multiplexing (OFDM), is the most promising technology to meet the requirement as a mechanism in rapid development of digital signal processing techniques.\\

The first introduction of OFDM dated back 1960s as a parallel data transmitting scheme. There are many realization proposals although the foundations are fixed generally. The basic idea is to divide a single high rate data stream into a number of lower-rate data 
streams. Each of these data streams is modulated on a specific carrier, which is called 
subcarrier, and transmitted simultaneously. Robustness will be preserve against multipath fading effect. Moreover, spectrum efficiency is enhanced in comparison to conventional multi-carrier transmission. OFDM considered as a frequency division multiplexing (FDM) where the data stream carried by each sub-carrier separated.\\

Traditional methods used in single-carrier modulation require a number of sinusoidal subcarrier oscillators in the modulator side and multipliers and correlators in the demodulator. Introduction of Discrete Fourier Transform (DFT) until 1971 made a revolution in the complexity development. The DFT block simplified the two side processes and helped to implement the baseband in the digital manner.\\
Since 1990s, OFDM has been employed in wideband data transmission. Applications of OFDM technology include asymmetric digital subscriber line (ADSL), high-bit-rate digital subscriber line (HDSL), and very-high-speed digital subscriber line (VDSL) in wired systems, and digital audio broadcasting (DAB), digital video broadcasting (DVB) in wireless systems. Furthermore, it has also been recognized as the basis of the wireless local area network (WLAN) standards, among which the IEEE 802.11a standard is one of the most important ones.\\

Two main topics in wireless and mobile communications are high data rate and high QoS, which cause communication systems be adaptive to fast varying channel conditions and providing a steady environment to various kinds of users at a high speed of data transmission.\\

Due to its capabilities of providing high data rate and less sensitivity to fast channel fading, OFDM technology, in combination with other powerful techniques such as the multiple-input, multiple-output (MIMO) technique, has been the mainstream of wireless and mobile networks. It has been used in various applications, such as wireless fidelity (Wi-Fi), worldwide inter-operability for microwave access (WiMAX), and the third generation partnership project (3GPP) long term evolution (LTE).\\

Recent development of digital integrated circuits, the high flexibility and low complexity of digital implementation of OFDM modem has accelerated its application. In competition of the technologies, field programmable gate array (FPGA) has attracted the most attention in recent years due to its superior performance and high flexibility. As a flexible general-purpose technology, FPGA is an array of gates that can be reconfigured by the designer as a versatile design platform. It is developed based on the programmable logic devices (PLDs) and the logic cell array (LCA) concept. By providing a two-dimensional array of configurable logic blocks (CLBs) and programming the interconnection that connects the configurable resources, FPGA can implement a wide range of arithmetic and logic functions. Compared to other popular IC technologies such as application specific integrated circuits (ASICs) and digital signal processors (DSPs), FPGA has the following advantages:\\

\textbf{Performance:} Inherently parallel architecture, FPGA has the ability to overcome the speed limit of sequential execution technologies and is able to process data at a much higher speed than DSP processors and whose performance is estimated by the system clock rate. Therefore, it can achieve much higher performance in various applications that requires large arithmetic resources, such as OFDM. However, DSP processors are still developing as an alternative.\\

\textbf{Reliability:} The high isolation and high parallelism mechanism not only minimize the reliability concerns, but also reduce the deterministic hardware dedicated to every task. Besides, there are mechanism in testing and verification of the system dynamically which are developing exponentially.\\

\textbf{Cost:} Because of its re-programmable nature, FPGA is a cost-effective solution for system development although the purchase cost are normally more than DSP processors which the architecture is fixed. It can be easily customized and reconfigured so that effectively versatile functionality can be realized using FPGA and there is no need to kick off design for each application. Normally, the products are tested and design and implement on the FPGA initially and after successful output it worth to transfer design into ASIC for mass production.\\

\textbf{Flexibility:} The most prominant functionality of FPGA is that the design can be changed rapidly in the prototype process. Recently, some other options like partial reconfiguration let the designer to look for more dynamic mechanisms. This let the manufacturers to have better performance in the time to market issues.\\
There is also some trends like IP core programs which help the big short-cuts in the designs but it very depends on the initial cost which should be decided very carefully.


\section{Motivation}

Practically, the signal is attenuated and distorted by multipath effect in real channel transmission. Fading estimation and equalization of the channel in wireless technology is inevitable to have a reliable communication. Implementation an OFDM system on FPGA with capability of channel estimation and synchronization is the final goal of this thesis.\\
There are many techniques and mechanism to implement OFDM wireless communication on FPGA. In ... the authors helped OFDM transceiver on certain topics in the receiver design, such as the synchronization, packet detection, channel estimation and equalization. Moreover, OFDM transceivers are designed for the AWGN channel have been presented in ..... However, there are not a comprehensive work presenting a complete development of OFDM system with channel estimation and synchronization using the FPGA technology.\\

A top-down approach and demonstrative system performance in baseband OFDM is done in this thesis. System synchronization will also be discussed in this thesis. In addition, we focus on the design and implementation of channel estimation and equalization, while a verification at system level is performed. The detailed objectives include:\\
\begin{itemize}
\item To design, model and implement after proper simulation a baseband OFDM system including both the transmitter and the receiver, and to analyze the system performance.
\item To prototype an OFDM system based on a specific wireless communication standard.
\item To implement the synchronization and channel estimation system for the receiver and provide system evaluation under different channel conditions.
\end{itemize}


\section{Methodology}
It is tried to explain the theoretical concepts firstly and then to show some facts in the simulation based on extracted models. Finally, the issues is examined on hardware.\\
The hardware chosen is consisted a Zynq board which is an FPGA with two embedded ARM processors and the radio board which is FMCOMMS1.

\section{Contribution}


\section{OFDM System Architecture}
Generally, an OFDM signal is defines as a summation of many OFDM standard symbols, which can considered continuous in the time domain.It can be defined as following:\\

\begin{equation} \label{general_form}
\begin{split}
s(t) =\sum\limits_{k=-\infty}^{+\infty} s_{k}(t)
\end{split}
\end{equation}

where $s_{k}(t)$ is the k-th OFDM symbol which starts at time $t= t_{s}$. An OFDM system is a multi-carrier transmission mechanism which the mathematical model is generalized by the summing a series of modulated subcarriers digitally. This modulation can be phase shift keying (PSK) or quadrature amplitude modulation (QAM) and transmitted in parallel. So, we can  conclude:\\
\begin{equation} \label{ofdm_digital_mod}
\begin{split}
s_{k}(t)=
\left\{
	\begin{array}{ll}
	Re\left( \sum\limits_{i=-\frac{N}{2}}^{\frac{N}{2}-1} d_{i+\frac{N}{2}} \exp\lbrack j2\pi(f_{c} - \frac{i+0.5}{T})(t- t_{s})\rbrack\right) , & t_{s}\le t < t_{s} + T\\
	0, & \mbox{otherwise}
	\end{array}
\right.
\end{split}
\end{equation}

where $T$ is the symbol duration, $N$ is the number of subcarriers, $f_{c}$ is the signal carrier
frequency on the radio frequency (RF) band, and $d_{i}$ is the complex value for PSK or QAM modulated symbol. We reach $I_{i}$ and $Q_{i}$ being the in-phase and quadrature part of $d_{i}$, respectively.\\
The complex envelope of an OFDM signal given by the following equation is used as the baseband notation:
\begin{equation} \label{ofdm_complex_env}
\begin{split}
s_{k}(t)=
\left\{
	\begin{array}{ll}
	Re\left( \sum\limits_{i=-\frac{N}{2}}^{\frac{N}{2}-1} d_{i+\frac{N}{2}} \exp\lbrack j2\pi\frac{i}{T}(t- t_{s})\rbrack\right) , & t_{s}\le t < t_{s} + T\\
	0, & \mbox{otherwise}
	\end{array}
\right.
\end{split}
\end{equation}

The real and imaginary parts of \ref{ofdm_complex_env} are the in-phase ($I$) and quadrature ($Q$) of the baseband OFDM signal. Consequently, they are multiplied by a cosine and a sine waveform with a carrier frequency to generate the passband OFDM. At the receiver, each subcarrier is down-converted with a subcarrier of the desired frequency and supported over the symbol period. For example, the complex value for the $m$-th subcarrier $d_{m}$ is obtained equation \ref{m_th_carrier}, where the whole signal is multiplied by the frequency of $\frac{m}{T}$, and then integrated over the symbol period $T$:
\begin{equation} \label{m_th_carrier}
\begin{split}
\int\limits_{t_{s}}^{t_{s} + T} s_{k}(t) \exp \lbrack -j2\pi \frac{m}{T} (t-ts) \rbrack dt & = \int\limits_{t_{s}}^{t_{s} + T}  \left( \sum\limits_{i=-\frac{N}{2}}^{\frac{N}{2}-1} d_{i+\frac{N}{2}} \exp\lbrack j2\pi\frac{i}{T}(t- t_{s})\rbrack\right) \exp \lbrack -j2\pi \frac{m}{T} (t-ts) \rbrack dt\\
& = \sum\limits_{i=-\frac{N}{2}}^{\frac{N}{2}-1} d_{i+\frac{N}{2}} \int\limits_{t_{s}}^{t_{s} + T}  \left( \exp\lbrack j2\pi\frac{i-m}{T}(t- t_{s})\rbrack\right) dt\\
& = Td_{m+ \frac{N}{2}}
\end{split}
\end{equation}

\ref{m_th_carrier} shows all the subcarriers over the integral region are zero except the desired one. The desired output for the signal demodulation, $d_{m+ \frac{N}{2}}$ multiplied to a constant factor $T$, is exactly the integration for the $m$-th subcarrier. Since each subcarrier has an exact integer number of cycles within OFDM symbol duration, the orthogonality between subcarriers is guaranteed.\\
the mathematical model for discrete time signal is as below if the OFDM symbol is sampled with a sampling period $\frac{T}{N}$:
\begin{equation} \label{math_model}
\begin{split}
s(n)= s_{k}(\frac{nT}{N})= \sum\limits_{i=0}^{N-1} d_{i} \exp(j2\pi\frac{in}{N}), n= 0, 1, ... , N-1
\end{split}
\end{equation}

This represents an inverse DFT (IDFT) for PSK or QAM symbols.\\
According to the above analysis, the basic architecture for a baseband OFDM system that contains the essential parts is shown in Figure~\ref{fig:basic_ofdm}.\\

\begin{center}
\includegraphics[width=12cm]{content/fig/basic_ofdm_design.JPG}
\label{fig:basic_ofdm}
\captionof{figure}{Basic baseband OFDM system}
\end{center}

In practice, to prevent sharp transactions at the sample time boundaries, a windowing block is used for filter shaping. In conclusion, spectrum utilization is enhanced dramatically. Therefore, the baseband OFDM symbol can be written as below:\\
\begin{equation} \label{ofdm_window}
\begin{split}
s_{k}(t)=
\left\{
	\begin{array}{ll}
	w(t - t_{s})\sum\limits_{i=-\frac{N}{2}}^{\frac{N}{2}-1} d_{i+\frac{N}{2}} \exp\lbrack j2\pi\frac{i}{T}(t- t_{s}- T{g}) , & t_{s}\le t < t_{s} + (1+\beta)T_{sym}\\
	0, & \mbox{otherwise}
	\end{array}
\right.
\end{split}
\end{equation}

where $T_{g}$ is the guard interval duration, $T_{sym}= T+ T_{g}$ is the OFDM symbol period,
symbol starting time $t_{s}= kT_{sym}$, and $w(t-t_{s})$ is the pulse shaping window, which is
usually a raised cosine filter, and $\beta$ is the roll-off factor.\\


\section{OFDM Specifications in IEEE 802.11a Standard}
\subsection{System Design}

In reality, having an anti-aliasing configuration, oversampling is performed before passing the digital signal to digital-to-analog converter. There are many other blocks in standards like channel coding, symbol interleaving and channel estimation.\\
In comparison to the fundamental architecture shown in Figure 000, some other building blocks are added in a practical IEEE 802.11 design shown in Fig 00, marked with blue and dashed on line. At the transmitter, several "null" subcarriers or tones are reserved besides of the data subcarriers in order to perform oversampling of the transmitted signal. In this context, "null" means the symbol carried on this subcarrier has a value of zero. Besides, some other subcarriers used as pilot for channel estimation are also inserted. The subcarriers are allocated at the input of the IFFT block to generate a phase shift. The windowing for pulse shaping is achieved after CP extension.\\

\begin{center}
\includegraphics[width=12cm]{content/fig/ieee_system_design.JPG}
\captionof{figure}{Architecture of an OFDM system}
\end{center}

At the receiver side, the frame synchronization and detection for both timing and frequency is performed in the first stage. Channel estimation is performed after the FFT block outputs the preambles in the frequency domain. The result is fed back to the FFT block for the equalization, which eliminates the effects of fading channel, while the fine synchronization for both timing and frequency is also added to further improve the system performance.\\
At least these basic parameters should be specified for a system design:\\
1) Delay spread expected for the channel (300 ns)\\
2) Guard duration (800 ns) which describes symbol duration (4.0 $\mu$s)\\
3) Available bandwidth\\
4) Data rate\\

For indoor environment a delay spread less than 300 ns expected. We consider the guard duration 800 ns, which effectively protects the signal from ISI in the indoor environment and some of the outdoor wireless communication environments. Five times the guard duration for limiting the power and bandwidth loss is regarded for the symbol duration, and is set to 4.0 $\mu$s in our case. Hence, the OFDM symbol rate is 0.25 mega symbol per second (Mbaud).\\
Keep in mind, the useful OFDM symbol duration without the guard interval is 3.2 $\mu$s. So, the subcarrier spacing, which is the reciprocal of the useful symbol duration, can be determined as 312.5 kHz. Assuming that there is a bandwidth of 20 MHz available, the number of subcarriers is calculated to be 64. This is exactly the same as the specification defined in IEEE 802.11a standard.\\
As mentioned, some tones are reseved for pilot subcarries (channel estimation), null subcarriers (realizing oversampling to avoid aliasing) and windowing (reduce the out-of-band spectral energy).\\
In our design we chose 48 data tones and 4 pilot subcarriers. So, 52 subcarries are occupied. Applying a raised-cosine window with roll-off factor $\beta$= 0.02 the total occupied bandwidth is\\
\begin{center}
$(1+ 0.02)\times(52 \times 312.5kHz) \approx 16.6MHz$
\end{center}

To accomplish Oversampling, some zeros before and after the data vector are appended in the frequency domain as shown below.\\

\begin{center}
\includegraphics[width=12cm]{content/fig/oversample.JPG}
\end{center}

The nonzero data values are mapped onto the subcarriers around 0 Hz, and the zeros are mapped onto frequencies around sampling rate.

Basically, in the BPSK modulation is applied on each subcarrier, each symbol for an
individual subcarrier has one bits. The bit rate achieves without channel coding:\\
\begin{center}
$\frac{1}{4.0\mu s} \times 48 \times 1= 12Mbps$
\end{center}

The same calculation can be perform for QPSK to reach 24Mbps. But, channel coding will reduce this values. Variation of coding rates and modulation methods, In the 802.11a standard, the data rate ranges from 6 Mbps to 54 Mbps.\\

\begin{center}
\captionof{table}{System parameters defined for the proposed OFDM system}
\vspace{0.5cm}
\begin{tabular}{c|c|c}
Parameter&Description&Value\\ \hline
$B_{w}$&Available channel bandwidth&$20MHz$\\
$\sigma_{\tau}$&Delay spread of the channel& $<300 ns$\\
$T_{g}$&Guard interval duration (Cyclic Prefix)&$0.8\mu s$\\
$T_{sym}$&OFDM symbol period&$4.0\mu$\\
$T$&Effective symbol duration (FFT period)&$3.2\mu s (=T_{g}- T_{sym})$\\
$\Delta f$&Subcarrier spacing&$312.5kHz (=1/T)$\\
$N_{g}$&Number of guard samples&$16$\\
$N$&FFT size&$64= B/\Delta f s$\\
$N_{d}$&Number of data subcarriers&$48$\\
$N_{p}$&Number of pilot sucarriers&$4$\\
$N_{u}$&Number of used subcarriers&$52$\\
$B_{u}$&Signal occupied bandwidth&$16.6MHz$\\
&Modulation type&$BPSK, QPSK$\\
$R_{b}$&Data rate without coding&$12Mbps, 24Mbps$\\
\end{tabular}
\end{center}

\subsection{IEEE 802.11a Standard in Time and Frequency}
A packet of OFDM will be described here. In an OFDM frame, a preamble which carries no data is transmitted first, followed by the signal field which give some information about data and transmitted data. As indicated in Figure 00, an OFDM frame has the general form as below:
\begin{center}
$s_{OFDM}(t)= s_{preamble}(t)+ s_{signal}(t- T_{preamble})+ s_{data}(t- T_{preamble}- T_{signal})$\\
\end{center}
where\\
\begin{center}
$s_{preamble}(t)= s_{short}(t)+ s_{long}(t- T_{short})$\\
\end{center}

The preamble starts with 10 short training symbols (STSs) from T1 to T10, followed by a guard interval (GI2) and two long training symbols (LTSs) L1 and L2. Both the short and long training sequences have an 8 $\mu$s duration and the entire preamble lasts for 16 $\mu$s. Then, a 3.2 $\mu$s signal symbol, as well as 800 ns guard interval is transmitted. This field bears some information necessary for the data symbols, such as the coding rate and length. Finally the various data symbols that carry user information are transmitted. Each data symbol has a duration of 4.0 $\mu$s, within which there is a 800 ns CP, as already described.\\
The application of STS and LTS for training are different. STS used for AGC, frame detection, coarse timing and frequency synchronization. Each symbol in this sequence has a duration of 800 ns and contains 16 samples, and is identical to one another.
 It will be shown in a professional system, auto-correlation will apply to this portion to perform such the operations. After the short training sequence is transmitted, a 1.6 $\mu$s guard interval that contains 32 samples is introduced. The LTS is cyclically extended within this interval. Then two identical LTSs with the same duration of 4.0 $\mu$s are followed. The LTS is used for fine frequency offset and channel estimation. It will be described that a cross-correlation with a stored array is done for extraction of the offset.\\



\begin{center}
\includegraphics[width=12cm]{content/fig/ofdm_frame.JPG}
\end{center}

As we already analyzed, 52 subcarriers are used for an OFDM data symbol and pilot. Oversampling is achieved by adding 12 null subcarriers in order to eliminate aliasing which might occur during digital to analog conversion. Because FFT shift is performed, the null subcarriers with a value of zero are located in the middle of the input vector for the IFFT block. Note that dc carrier is not used to transmit data. The short and long training sequences can also be applied to this mapping rule, since they both have a length of 52 samples with frequency index from -26 to +26.\\


\section{Hardware Introduction}
\subsection{Processor}
\begin{center}
\includegraphics[width=12cm]{content/fig/ZC706.JPG}
\captionof{figure}{Xilinx Zynq-7000 SoC ZC706 Evaluation Kit}
\end{center}

\subsection{Radio Board}
\begin{center}
\includegraphics[width=10cm]{content/fig/fmcomms1.jpg}
\captionof{figure}{AD-FMCOMMS1-EBZ (Radio Board)}
\end{center}

\begin{center}
\includegraphics[width=15cm]{content/fig/fmcomms1Blockdiagram.jpg}
\captionof{figure}{AD-FMCOMMS1-EBZ Block Diagram}
\end{center}

\subsection{Clock Chain on FMCOMMS1}
Now, we discuss more about the clock chain and distribution mechanism on the board to find some meaningful number. As you can see in the figure, we configure the board and internal FPGA architecture to generate a 30MHz clock to the RF board. This 30MHz is just chosen because a relevant crystal mounted on the Zynq board and the all generated clock is supposed to be in-phased with it. This 30MHz is an input for AD9548 as a clock generator/synchronizer which has a very precise PLL inside to generate a 20MHz.\\
The AD9548 generates an output clock synchronized to one of up to four differential or eight single-ended external input references. The digital PLL allows for reduction of input time jitter or phase noise associated with the external references. The AD9548 continuously generates a clean (low jitter), valid output clock even when all references have failed by means of a digitally controlled loop and holdover circuitry. AD9548 is a very complicated device to generate 20MHz with maximum precision.\\

\begin{center}
\includegraphics[width=15cm]{content/fig/ad9548BlockDiagram.JPG}
\captionof{figure}{AD9548 Block Diagram}
\end{center}

The next IC in the clock chain is AD9523-1 which is Low Jitter Clock Generator. The AD9523-1 provides a low power, multi-output, clock distribution function with low jitter performance, along with an on-chip PLL and VCO with two VCO dividers. The on-chip VCO tunes from 2.94 GHz to 3.1 GHz. The AD9523-1 is defined to support the clock requirements for
long term evolution (LTE) and multicarrier GSM base station designs. It relies on an external VCXO to provide the reference
jitter cleanup to achieve the restrictive low phase noise requirements necessary for acceptable data converter SNR performance.\\
The input receivers, oscillator, and zero delay receiver provide both single-ended and differential operation. When connected to a recovered system reference clock and a VCXO, the device generates 14 low noise outputs with a range of 1 MHz to 1 GHz, and one dedicated buffered output from the input PLL (PLL1). The frequency and phase of one clock output relative to another clock output can be varied by means of a divider phase select function that serves as a jitter-free, coarse timing adjustment in increments that are equal to half the period of the signal coming out of the VCO.\\
In  our chain we have a 80MHz VCXO connected to AD9523-1. It is supposed to generated 40MHz for ADC, DAC and also the main OFDM architecture FPGA program. You can see the specification of the crystal oscillator in ....\\

\begin{center}
\includegraphics[width=10cm]{content/fig/cvhd.JPG}
\captionof{figure}{CVHD-950 Ultra Low Phase Noise Oscillator}
\end{center}


\section{Expectations for CFO on FMCOMM1}

In implementation of an OFDM chain, we should have good understanding the range of carrier frequency offsets which can be expected on our hardware platform. Our RF board foundation is based on FMCOMMS1. As a result, the elements in term of phase -noise and CFO should be studied. The main RF frequency refrence is a Crystek CVHD-950 (VCXO). This VCXO provides a clock signal at a nominal frequency of 80 MHz. Actual output frequency varies as a function of multiple factors, and is only specified by the manufacturer with some tolerance. The CVHD-950 is specified with a frequency tolerance of $\pm$4 ppm. Thus, we must design for a reference frequency of 80$\pm$0.000320 MHz. Imagine our target RF carrier frequency is 2452 MHz which implies 2400 MHz$\pm$4 ppm (or 2400$\pm$0.009600 MHz).\\
The worst case CFO will occur when the transmit and receive nodes operate at opposite ends of this range. Thus, for operation in the 2.4 GHz band our OFDM transceiver design must be ready to handle any carrier frequency offset up to $\approx$20 kHz.


\section{Time Domain CFO Correction}
Prevention of the degradation of CFO, the receiver should estimate and correct the offset in the time domain before the FFT block. The FFT block translates the received signal into the frequency. Regarding to the variety issue of OFDM, many estimation algorithms have been proposed.\\

\section{System Analysis by Simulation}
\label{sec_anasim}

\section{Carrier Frequency Offsets}
\label{sec_simstruct}
As a result of the frequency variation between local oscillators of the transmitter and the receiver nodes that generate the carrier signals, carrier frequency offsets (CFO) is happened. It causes when the baseband signal is going to be translated to RF. The issue is understood well but the impact to overcome CFO and suppression this phenomena is always depend on the specific parameters of the given transceiver and the hardware.\\
The origin of the CFO effect is studied in this section. We explore in a specific scenario of OFDM and the impact on the hardware design. Both simulation and experiments will be demonstrated and the CFO estimation and compensation is described.\\

\subsection{Origin of CFO}
A simple model of a radio transmitter and receiver can depict the basis of the CFO source.

\begin{center}
\includegraphics[width=\textwidth]{content/fig/cfo_radio_model.JPG}
\captionof{figure}{General models of a direct conversion RF}
\label{cfo_radio_model}
\end{center}

In Figure \ref{cfo_radio_model}, $\omega$ is The carrier frequency and $X_{BB}$ is the complex baseband signal, $X_{RF}$ is
a real-valued RF signal. These models simplified many other operations in a real RF transceivers although none of these affect the up/down-conversion processes as they relate to CFO.\\
These equations about transmit and receive processes can be written in equation (\ref{eq_tx_rx_process}):

\begin{equation}\label{eq_tx_rx_process}
\begin{split}
X_{RF} & = TX(X_{BB})\\
& = Re(X_{BB})\cos(\omega t) - Im(X_{BB})\sin(\omega t)\\
& = \frac{1}{2} (X_{BB} e^{jt\omega} + X^{*}_{BB} e^{-jt\omega})\\
\\
X_{BB} & = RX(X_{RF}, \omega)\\
& = LPF(X_{RF} e^{j\omega t})
\end{split}
\end{equation}


Assume a signal $S_{BB}$ transmitted with carrier frequency $\omega_{S}$ which is received with carrier frequency $\omega_{D}$. we can express the received baseband signal $D_{BB}$ in terms of the transmitted baseband signal $S_{BB}$ and the carrier frequencies. Then:\\

\begin{equation} \label{DBB_SBB}
\begin{split}
D_{BB} & = LPF(\frac{(S_{BB} e^{jt\omega_{S}} + S^{*}_{BB} e^{-jt\omega_{S}})e^{jt\omega_{D}}}{2})\\
&= S_{BB}(e^{jt(\omega_{S}- \omega_{D})})
\end{split}
\end{equation}

The received baseband signal is equal to the original baseband signal modulated by a complex sinusoid. In the frequency domain, this gives a received spectrum equal to the transmitted one, only shifted away from DC by the difference in the carrier frequencies of the transmitter and receiver (i.e. $\omega_{S}- \omega_{D}$). This shift of the received
signal is the baseband manifestation of carrier frequency offset.

\subsection{Impact of CFO}
\label{Impact_of_CFO}
There are two destructive impacts on an OFDM system. Firstly, the phase offset across subcarriers in an symbol which can be  estimated and corrected in frequency domain to prevent errors in a constant rotated constellation. Some subcarriers are allocated as pilot tones which receiver can estimate phase errors.\\
The second effect of CFO is the degradation of orthogonality between subcarriers in receiver's FFT which causes inter-carrier interference (ICI). ICI acts an effective SNR reduction as a result of CFO increasing. [...]\\
The impact is displayed in Figure \ref{cfo_impact_on_ici} which is shown simulated OFDM system uses 10 MHz bandwidth and 64 subcarriers, 48 of on a random 16-QAM data symbols. CFO and AWGN are applied between the transmitter and receiver. The receiver model uses perfect knowledge of the CFO to correct the phase offset in each OFDM symbol, but does not implement any correction for ICI.\\

\begin{center}
\includegraphics[width=\textwidth]{content/fig/cfo_on_ici.JPG}
\captionof{figure}{OFDM performance loss due to CFO-induced ICI.}
\label{cfo_impact_on_ici}
\end{center}

The results shows that for large CFOs errors caused by ICI dominate performance, even at high SNR. It is also
clear that for small CFOs performance is dominated by SNR. Specifically, for frequency offsets smaller than 1 kHz, the
performance degradation due to ICI is negligible.\\



\chapter{FPGA Implementations}
\label{sec:chapter_3}

The hardware is introduced with some details of the implementations. The main FPGA side project is done in Xilinx System Generator which is a high level alternative with standard scripting languages like VHDL and Verilog. An overview to select the radio board and the clock chain inside will be described.\\
Main blocks in transmitter and receiver is defined and the mechanism for packet detection is illustrated in details.

\section{System Design in System Generator}
\label{sec_anasim}

Simulink® from The MathWorks® is a powerful graphical modeling system which allows complex systems to be designed using a block diagram methodology. Xilinx System Generator for DSP is a blockset for Simulink® which allows the modeling of fixed point systems which can be transformed into VHDL and targeted at an FPGA. Automatic generation of the bitstream is supported with the synthesis and implementation tools run from within the Simulink® environment.\\

\begin{figure}[h!]
\centering
\includegraphics[width=10cm]{content/fig/systemGen.JPG}
\caption{System Generator Cycle.}
\label{fig:systemGen}
\end{figure}

Then main core of an an OFDM modulator and demodulator are the inverse FFT (IFFT) and FFT respectively. In 802.11a WLAN standard a 64-point transform with 52 of the subcarriers are carrying user data in a BPSK,
QPSK, 16-QAM or 64-QAM alphabet. The symbol rate in this systems is $20 MSym/s$. The OFDM symbol period
is $4 \mu s$, with $3.2 \mu s$ of this interval occupied by the 64-point FFT symbol and the additional $0.8 ps$ used for the cyclic prefix.\\

Figure \ref{ofdm_system} shows the main scheme of an OFDM system in transmitter and receiver. Another block of Channel is a Additive White Gaussian Noise which is used in simulation only.\\
As you can see there are some others blocks which are necessary for system implementations. The whole system is connected to a main hard processor which is located in the FPGA. It is a ARM Cortex-A9 with maximum frequency of $666.66 MHz$. EDK processor represents the main processor which connected to the OFDM block by AXI protocol. 

\begin{figure}[h!]
\centering
\includegraphics[width=10cm]{content/fig/system.JPG}
\caption{OFDM System..}
\label{fig:ofdm_system}
\end{figure}


Figure \ref{tx_block} illustrates the transmitter which consists of many blocks. Controlling of the time we have \textit{TxControl} block for synchronization of the blocks. It also generate the semi-fixed preamble (LTS and STS) and the relevant Training signals for the system. \textit{Training Data} generates the training pattern which is used to estimate the channel frequency response.
The main clock is IFFT which convert the time-based signals into frequency. In the current picture we set a 64 point IFFT although the recent design it upgraded to 256 as a result of the strategy changes. The data captured by the IFFT blocks are integrated in the \textit{OutputBuffers}. \textit{OutputMuxes} block chooses between two possible antenna to transmit the stream. In \textit{PreSpin, Filters DACs}, some sub-blocks for soft gain and DAC preparation data are implemented. Besides, the are a generic block to rate change matter which can be activated by the processor.\\

\begin{figure}[h!]
\centering
\includegraphics[width=\textwidth]{content/fig/txblock.JPG}
\caption{OFDM Transmitter Block.}
\label{fig:tx_block}
\end{figure}

Figure \ref{rx_block} demonstrates the receiver block with its main blocks. The input signals enter into the device in a I/Q form from the analogue board. The is \textit{ADC inputs Antenna Selection} which we can switch between the two antennas and also the internal TX block which is reside in to the FPGA for the testing purposes. This block also adjust the input gain for the rest of the design. The frequency correction is done in the \textit{Coarse Freq Correction} using the STS stream.\\

\begin{figure}
\centering
\includegraphics[width=\textwidth]{content/fig/rxblock.JPG}
\caption{OFDM Receiver Block.}
\label{fig:rx_block}
\end{figure}



In the \textit{Packet Detection} block an auto-correlation approach is done on the signal to detect the energy of the preamble in the beginning of STS shows in Figure \ref{autocorrblock}. This is implemented based on the magnitude square of the both I/Q signals and comparing with a threshold after a sliding window. In other branch a multiplication of of the imaginary and real part with their 16 clock delayed version is calculated and the square magnitude in a sliding windows is detected. In \textit{Detection Decision} we use some other threshold to be ensure if the two cross correlation peaks are detected in the right time to signal the packet detection.

\begin{figure}
\centering
\includegraphics[width=\textwidth]{content/fig/autocorrblock.JPG}
\caption{Auto-Correlation Block.}
\label{fig:autocorrblock}
\end{figure}

\begin{figure}
\centering
\includegraphics[width=\textwidth]{content/fig/fine_packetDetect.JPG}
\caption{Fine Packet Detection Block.}
\label{fig:fine_packetDetect}
\end{figure}

As described in Section \ref{section:channel_est}, the main part of the receiver is estimation of the channel which can be acquire by the LTS and training signal at the preamble. It is one of the most complex issue of the whole but the main part is shown in Figure \ref{cmpx_div} and \ref{division}.\\

\begin{figure}
\centering
\includegraphics[width=\textwidth]{content/fig/cmpx_div.JPG}
\caption{Complex Division Block.}
\label{fig:cmpx_div}
\end{figure}

\begin{figure}
\centering
\includegraphics[width=\textwidth]{content/fig/division.JPG}
\caption{Division Block.}
\label{fig:division}
\end{figure}


\section{Hardware Introduction}
\subsection{FPGA Board}

The ZC706 evaluation board for the XC7Z045 All Programmable SoC (AP SoC) provides a hardware environment for developing and evaluating designs targeting the Zynq-7000 XC7Z045-2FFG900C AP SoC. The ZC706 evaluation board provides features common to
many embedded processing systems, including DDR3 SODIMM and component memory, a four-lane PCI Express interface, an Ethernet PHY, general purpose I/O, and two UART interfaces. Other features can be supported using VITA-57 FPGA mezzanine cards (FMC) attached to the low pin count (LPC) FMC and high pin count (HPC) FMC connectors. For details of architecture see Section \ref{fpga_arch}.\\

\begin{figure}
\centering
\includegraphics[width=12cm]{content/fig/ZC706.JPG}
\caption{ZC706 Evaluation Borad Block Diagram.}
\end{figure}

\begin{figure}
\centering
\includegraphics[width=12cm]{content/fig/zc706_block_diagram.JPG}
\caption{Xilinx Zynq-7000 SoC ZC706 Evaluation Kit}
\label{fig:zc706}
\end{figure}

\subsection{Radio Board}

The AD-FMCOMMS1-EBZ high-speed analog module is designed to showcase one of the latest generation high-speed data converters. The AD-FMCOMMS1-EBZ provides the analog front-end for a wide range of compute-intensive FPGA-based radio applications.\\
The AD-FMCOMMS1-EBZ enables RF applications from 400MHz to 4 GHz. The module is customizable to a wide range of frequencies by software without any hardware changes, providing options for GPS or IEEE 1588 Synchronization, and MIMO configurations. When combined with the Xilinx ZC706, AD-FMCOMMS1-EBZ enables a variety of wireless communications functions at the physical layer, from baseband to RF. With up to 4 GB of flash storage space, 512 MB of RAM, Gigabit Ethernet interface (depending on the base platform). The platform offers enough flexibility for many applications, and supports streaming data, and standard web interfaces to analyze transmitted RF data.\\

\begin{figure}
\centering
\includegraphics[width=10cm]{content/fig/fmcomms1.jpg}
\caption{AD-FMCOMMS1-EBZ (Radio Board)}
\end{figure}

\begin{figure}
\centering
\includegraphics[width=15cm]{content/fig/fmcomms1Blockdiagram.jpg}
\caption{AD-FMCOMMS1-EBZ Block Diagram}
\label{fig:fmcomms1}
\end{figure}

\subsection{Clock Chain on FMCOMMS1}
Now, we discuss more about the clock chain and distribution mechanism on the board to find some meaningful number. As you can see in the figure, we configure the board and internal FPGA architecture to generate a 30MHz clock to the RF board. This 30MHz is just chosen because a relevant crystal mounted on the Zynq board and the all generated clock is supposed to be in-phased with it. This 30MHz is an input for AD9548 as a clock generator/synchronizer which has a very precise PLL inside to generate a 20MHz.\\
The AD9548 generates an output clock synchronized to one of up to four differential or eight single-ended external input references. The digital PLL allows for reduction of input time jitter or phase noise associated with the external references. The AD9548 continuously generates a clean (low jitter), valid output clock even when all references have failed by means of a digitally controlled loop and holdover circuitry. AD9548 is a very complicated device to generate 20MHz with maximum precision.\\

\begin{figure}
\centering
\includegraphics[width=15cm]{content/fig/ad9548BlockDiagram.JPG}
\caption{AD9548 Block Diagram}
\label{fig:ad9548}
\end{figure}

The next IC in the clock chain is AD9523-1 which is Low Jitter Clock Generator. The AD9523-1 provides a low power, multi-output, clock distribution function with low jitter performance, along with an on-chip PLL and VCO with two VCO dividers. The on-chip VCO tunes from 2.94 GHz to 3.1 GHz. The AD9523-1 is defined to support the clock requirements for
long term evolution (LTE) and multicarrier GSM base station designs. It relies on an external VCXO to provide the reference
jitter cleanup to achieve the restrictive low phase noise requirements necessary for acceptable data converter SNR performance.\\
The input receivers, oscillator, and zero delay receiver provide both single-ended and differential operation. When connected to a recovered system reference clock and a VCXO, the device generates 14 low noise outputs with a range of 1 MHz to 1 GHz, and one dedicated buffered output from the input PLL (PLL1). The frequency and phase of one clock output relative to another clock output can be varied by means of a divider phase select function that serves as a jitter-free, coarse timing adjustment in increments that are equal to half the period of the signal coming out of the VCO.\\
In  our chain we have a 80MHz VCXO connected to AD9523-1. It is supposed to generated 40MHz for ADC, DAC and also the main OFDM architecture FPGA program. You can see the specification of the crystal oscillator in ....\\

\begin{figure}
\centering
\includegraphics[width=10cm]{content/fig/cvhd.JPG}
\caption{CVHD-950 Ultra Low Phase Noise Oscillator}
\label{fig:cvhd}
\end{figure}


\section{Expectations for CFO on FMCOMM1}

In implementation of an OFDM chain, we should have good understanding the range of carrier frequency offsets which can be expected on our hardware platform. Our RF board foundation is based on FMCOMMS1. As a result, the elements in term of phase -noise and CFO should be studied. The main RF frequency refrence is a Crystek CVHD-950 (VCXO). This VCXO provides a clock signal at a nominal frequency of 80 MHz. Actual output frequency varies as a function of multiple factors, and is only specified by the manufacturer with some tolerance. The CVHD-950 is specified with a frequency tolerance of $\pm$4 ppm. Thus, we must design for a reference frequency of 80$\pm$0.000320 MHz. Imagine our target RF carrier frequency is 2452 MHz which implies 2400 MHz$\pm$4 ppm (or 2400$\pm$0.009600 MHz).\\
The worst case CFO will occur when the transmit and receive nodes operate at opposite ends of this range. Thus, for operation in the 2.4 GHz band our OFDM transceiver design must be ready to handle any carrier frequency offset up to $\approx$20 kHz.


\section{Time Domain CFO Correction}
Prevention of the degradation of CFO, the receiver should estimate and correct the offset in the time domain before the FFT block. The FFT block translates the received signal into the frequency. Regarding to the variety issue of OFDM, many estimation algorithms have been proposed.\\

\section{Carrier Frequency Offsets}
\label{sec_simstruct}
As a result of the frequency variation between local oscillators of the transmitter and the receiver nodes that generate the carrier signals, carrier frequency offsets (CFO) is happened. It causes when the baseband signal is going to be translated to RF. The issue is understood well but the impact to overcome CFO and suppression this phenomena is always depend on the specific parameters of the given transceiver and the hardware.\\
The origin of the CFO effect is studied in this section. We explore in a specific scenario of OFDM and the impact on the hardware design. Both simulation and experiments will be demonstrated and the CFO estimation and compensation is described.\\

\section{FPGA Architecture}
\label{fpga_arch}
Based on the Xilinx All programmable SoC architecture, the Zynq-7000 All Programmable SoCs enable extensive system level differentiation, integration, and flexibility through hardware, software, and I/O programmability. Using the Zynq-7000 platform, you can design smarter systems with tightly coupled software based control and analytic with real time hardware-based processing and optimized system interfaces.\\
As you can see in Figure \ref{zynq_inside} the foundation of Zynq-7000 is divided into two main parts. Firstly, Processor System which are two ARM processors and the fixed implemented peripherals. The rest are just the raw Programmable Logic which the main OFDM physical layer is implemented inside. We should build the gates, DSP and RAM in this region by VHDL programming or System Generator software in Matlab environment.\\

\begin{figure}
\centering
\includegraphics[width=12cm]{content/fig/zynq_inside.JPG}
\caption{Zynq-7000 Diagram}
\label{fig:zynq_inside}
\end{figure}

The block diagram of the design is illustrated in Figure \ref{design_block_diagram}. We activate one of the ARM processors. In the TX chain, a PC sends a packet data via the Ethernet port to Zynq-7000. The EMAC block receives the packet and DMA it into the RX Block RAM which is divided into $32$ bank with size of $2K \times 64-bit$ which realize a Circular Buffer to relief the burst data stream enters from the asynchronous Ethernet port. As the OFDM block works in $40 MHz$ and each $2K$ block reading takes maximum $50\mu s$ for the EMAC and OFDM-PHY pessimistically, the tolerance of the Ethernet stream will be $\dfrac{32 \times \ 64b}{50\mu s} = 40Mbps$ which proves good number of banks. The offset pointer of reading and writing by DMA MAC which is govern under a scatter-gather scheme and the OFDM-PHY is controlled by the ARM.\\

\begin{figure}
\centering
\includegraphics[width=12cm]{content/fig/fpga_internal.JPG}
\caption{Design Block Diagram}
\label{fig:design_block_diagram}
\end{figure}

As you can see in the block diagram the connection bridge between the Processor System and the Programmable Logic in the Zynq architecture can be AXI protocol. You can find the detail of AXI at ref..... There are some other communication protocols but in the Zynq design AXI works optimum.
For easier programming issues in PC side, we used Linux Virtual Machine inside a Windows OS. It is very helpful because this configuration prevents unnecessary data exchange of the system and helps us to have a real estimation of the bit rate.\\

\section{Test Methodology}
The configuration set-up is consisted of two Zynq board each carrying a FMCOMMS1 radio board. They are connected to two individual PC via Ethernet cables as shown in Figure \ref{hardware_setup}.\\

\begin{figure}
\centering
\includegraphics[width=\textwidth]{content/fig/hardware_setup.JPG}
\caption{Hardware set-up}
\label{fig:hardware_setup}
\end{figure}

This configuration should be tested partially and have realistic estimation of the maximum possible bit-rate and then calculate SNR of channel. Having engineering steps to examine each hardware block, we check the loops illustrated in \ref{design_block_diagram}.\\
The maximum bit rate we could reach to exchange via Ethernet peripheral of the Zynq board which is called EMAC is $600 Mbps$. This was a time consuming task to reach to this bit rate considering the complicated Direct Memory Access mechanism implemented in near contact of EMAC inside of ARM processor. Fortunately, there were many useful application examples dedicated by Xilinx but still it should study many document to understand the scheme in RX and TX of Ethernet.\\
Next, the correct configuration of the two TX and RX BRAMs are checked by directly data replacement between the two banks specified by the ARM processor. This was an important step because the maximum data rate is very depends on the correct data reading and writing into these two blocks. There is an useful functionality of the RAM which is designed also in Zynq-7000 that called Error-Correction Code (ECC). This is a type of computer data storage that can detect and correct the most common kinds of internal data corruption. ECC memory is used in most computers where data corruption cannot be tolerated under any circumstances, such as for scientific or financial computing. The two BRAMs communicate with AXI Bus via an BRAM controllers. The configuration of the BRAMs and the controllers should be set accordingly.\\
The main part of the project is dedicated of the OFDM-PHY block with its sophisticated details.\\




\begin{abstract}
An OFDM transmitter and receiver is implemented on FPGA.
\end{abstract}

\section{Introduction}


\section{Simulation Structure}
\label{sec_simstruct}


% References
\cleardoublepage

\bibliographystyle{unsrt}
\bibliography{Thesis_Bibliography}	


%\newpage
%\appendix
%\appendixpage
%\chapter{Use cases}\label{appexdix:usecases}

%The appendix provides the representation 

\end{document}
